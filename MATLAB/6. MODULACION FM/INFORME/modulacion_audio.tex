\documentclass[conference]{IEEEtran}
\IEEEoverridecommandlockouts
% The preceding line is only needed to identify funding in the first footnote. If that is unneeded, please comment it out.
\usepackage{cite}
\usepackage{amsmath,amssymb,amsfonts}
\usepackage{algorithmic}
\usepackage{graphicx}
\usepackage{textcomp}
\usepackage{xcolor}
\usepackage{tabularx}
\usepackage{multirow}
\usepackage{graphics} % for pdf, bitmapped graphics files
\usepackage{subfig}
\usepackage{subcaption}
\usepackage{hyperref}
\usepackage{academicons}
\usepackage{xcolor}
\usepackage{listings}
\usepackage{tabularx} % Asegúrate de incluir este paquete

\usepackage{tikz}
\usetikzlibrary{shapes.geometric, arrows}

\usetikzlibrary{shapes.geometric, arrows}

\tikzstyle{startstop} = [rectangle, rounded corners, minimum width=3cm, minimum height=1cm,text centered, draw=black, fill=red!30]
\tikzstyle{process} = [rectangle, minimum width=3cm, minimum height=1cm, text centered, draw=black, fill=blue!30]
\tikzstyle{arrow} = [thick,->,>=stealth]


\def\BibTeX{{\rm B\kern-.05em{\sc i\kern-.025em b}\kern-.08em
		T\kern-.1667em\lower.7ex\hbox{E}\kern-.125emX}}

% Color Enlace
\definecolor{colorEnlace}{RGB}{0, 0, 0}
\hypersetup{
	colorlinks=true,
	linkcolor=colorEnlace,
	citecolor=colorEnlace,
	urlcolor=colorEnlace,
	pdfauthor={Davis Bremdow Salazar Roa},
	pdftitle={Sistemas Embebidos}
}
\definecolor{mybg}{rgb}{0.97,0.97,0.97}
\definecolor{mygray}{gray}{0.4}
\definecolor{mygreen}{rgb}{0,0.6,0}
\definecolor{myblue}{rgb}{0,0,0.8}
\definecolor{mypurple}{rgb}{0.58,0,0.82}
\definecolor{myred}{rgb}{0.7,0,0}

\lstdefinelanguage{MatlabEnhanced}{
	language=Matlab,
	morekeywords={[2]linspace,plot,title,xlabel,ylabel,legend,grid},
	morekeywords={[3]sin,cos,exp,log,sqrt},
	keywordstyle=\color{myblue}\bfseries,
	keywordstyle=[2]\color{mypurple},
	keywordstyle=[3]\color{myred},
	commentstyle=\color{mygreen}\itshape,
	stringstyle=\color{mygray},
	morecomment=[l]%
}

\lstset{
	language=MatlabEnhanced,
	backgroundcolor=\color{mybg},
	frame=single,
	basicstyle=\ttfamily\small,
	showstringspaces=false,
	numbers=none,              %
	xleftmargin=0pt,           %
	framexleftmargin=0pt,      
	framexrightmargin=0pt,
	framextopmargin=2pt,
	framexbottommargin=2pt,
	breaklines=true,
	tabsize=1,
}

% Control 
\usepackage{amsmath}
\begin{document}
	
	\title{Modulación FM - Análisis de una señal de voz}
	\author{
		\makebox[\textwidth][c]{\large\textbf{Universidad Nacional de San Antonio Abad del Cusco}}\\
		\makebox[\textwidth][c]{\normalsize\textit{Escuela profesional de Ingeniería Electrónica}}\\
		\makebox[\textwidth][c]{\normalsize\textit{Laboratorio de Circuitos Electrónicos III}}\\
		\and
		\IEEEauthorblockN{Ing. Milton Velasquez Curo}
		\IEEEauthorblockA{Ingeniero Electrónico \\
			Cusco, Perú \\
			milton.velasquez@unsaac.edu.pe}
		\and
		\IEEEauthorblockN{Ruth Juana Espino Puma - 185746}
		\IEEEauthorblockA{Estudiante de Ingeniería Electrónica \\
			Cusco, Perú \\
			184657@unsaac.edu.pe}
		\and
		\IEEEauthorblockN{Davis Bremdow Salazar Roa - 200353}
		\IEEEauthorblockA{Estudiante de Ingeniería Electrónica \\
			Cusco, Perú \\
			200353@unsaac.edu.pe}
	}
	
	\maketitle
	\begin{abstract}
		La modulación es un proceso fundamental en las telecomunicaciones que consiste en variar una o más propiedades de una señal portadora (como amplitud, frecuencia o fase) en función de una señal de información o mensaje. Este proceso permite transmitir información a largas distancias de manera eficiente, minimizando interferencias y aprovechando mejor el espectro de frecuencias. Existen varios tipos de modulación, entre ellos la modulación en amplitud (AM), frecuencia (FM) y fase (PM), cada una con características y usos específicos según las necesidades del sistema de comunicación.
		
	\end{abstract}
	
	\begin{IEEEkeywords}
		Modulación, portadora, señal, índice de modulación, amplitud, frecuencia, fase, transmisión, espectro, distorsión.
	\end{IEEEkeywords}
	
	%% Contenido del documento
	
	\section{ Modulación FM }
	Para realizar la modulación FM de una señal de voz existen varios caminos a tomar dentro de MATLAB dentro de los cuales se pueden apreciar la función fmmod o el uso de las herramientas integrales para definir la expresión FM resultante.
	
	Para este propósito se hizo uso del código mostrado en el listing \ref{lst:modulacion-fm} define una modulación mediante el empleo de la función $cumsum()$ la cual se encarga de integrar la señal moduladora.
	
	\begin{lstlisting}[numbers=none, caption="Modulacion FM de una señal de Audio", label=lst:modulacion-fm]
		%% Parametros de la senal de informacion
		fm = 15;
		a = 2;
		
		%%  Parametro de la portadora
		fc = 100;
		A = 5;
		
		Ts = 1/(1000*fc);
		Fs = 1/Ts;
		t = 0: Ts : 0.5 - Ts;
		
		ft = a*sin(2*pi*fm*t);
		
		%% Modulacion AM
		fi = A*cos(2*pi*fc*t + 2*pi*cumsum(ft)*Ts);
	\end{lstlisting}
	
	El agregado para este procedimiento se realiza mediante la definición de una variable que se encarga de recuperar la información de un archivo .wav en la cual se tiene registrado un audio de 2 segundos, en el listing \ref{lst:audio-matlab} se define estas líneas de código.
	
	\begin{lstlisting}[caption={Lectura del archivo del audio}, numbers=none, label=lst:audio-matlab]
		y = audioread("audio.wav");
	\end{lstlisting}
	
	En este pequeño fragmento de código y la variable $y$ almacenan 16000 muestras registradas para 2 segundos de grabación.
	
	Ya con los datos del archivo de audio (señal moduladora), esta se puede emplear para realizar la modulación de la señal de información mediante la portadora, sin embargo antes de ello es necesario conocer las componentes de frecuencia de la señal de audio para hacer uso de una portadora adecuada con la cual se pueda adecuar la señal de información para su traslado en frecuencia.
	
	Una primera aproximación en frecuencia para la señal de audio se puede obtener mediante la transformada de Fourier la cual nos brindará que componentes (tonos u otros) se encuentran más representado en este dominio.
	
	\begin{figure}[h]
		\centering
		\includegraphics[width=0.5\textwidth]{media/fft-audio}
		\caption{Transformada de Fourier - Audio}
		\label{fig:fft-audio}
	\end{figure}
	
	
	En la figura \ref{fig:fft-audio} se puede apreciar el espectro de la señal de audio en la cual se puede apreciar 3 tonos significativos en:
	\begin{itemize}
		\item $f_1 = 235 [Hz] $
		\item $f_2 = 505 [Hz]$
		\item $f_3 = 858 [Hz]$
	\end{itemize}
	
	Siendo necesario el empleo de una portadora superior a la máxima frecuencia obtenida para evitar comportamientos inesperados durante este proceso, considerando además que el proceso de modulación requiere del traslado de la señal en banda base a una mayor representación en frecuencia.
	
	\section{ Demodulación FM }
	
	Existen diferentes formas de poder recuperar la señal de información en una señal FM, sin embargo como se define en \cite{stremler2006} cada uno de estos métodos deben cumplir con una característica esencial que relaciona la amplitud de la señal de salida de lineal con la frecuencia instantánea de la señal de entrada.
	
	Siendo así que uno de estos métodos (directo), relaciona de forma lineal el voltaje en frecuencia, dandole el nombre de discriminador de frecuencias el cual consiste a grandes rasgos el empleo de un derivador para convertir la señal FM a una parecida a la FM y posterior a ello utilizar un detector de envolvente para recuperar la señal que ahora se encuentra en la amplitud de la señal.	
	
	\begin{equation}
		\frac{d \phi}{dt} = A \left[ \omega_c + k_f f(t) \right] sin \left( \omega_c t + \int_{0}^{t} f(\tau) \, d\tau \right)
		\label{eq:demod-fm}
	\end{equation}
	
	En \ref{eq:demod-fm} se muestra la expresión que se obtiene al derivar la expresión generar de una señal FM, obteniendo la señal de información como parte de la amplitud de la señal FM resultante, lo cual facilita la recuperación de la señal de información al aplicar un detector de envolvente a la señal derivada mediante el empleo de un circuito derivador $ s = jw $.
	
	\subsection{Demodulación FM en MATLAB}
	Una forma de poder representar este comportamiento de forma digital mediante el empleo de MATLAB es mediante el uso de la transformada de Hilbert que actúa a modo de detector de envolvente luego de aplicar la derivada a la señal FM modulada previamente.
	
	En el bloque de código \ref{lst:demod-fm} se muestra el código definido para este propósito la cual a su vez depende de la etapa previa moduladora.
	
	\begin{lstlisting}[numbers=none, caption="Demodulación FM", label=lst:demod-fm]
		%% Demodulacion FM
		ftq = hilbert(fi).*exp(-j*2*pi*fc*t); % Senal en cuadratura
		% diff es para realizar una derivada
		% unwrap es para evitar los saltos
		
		dftq = [ftq(1) (1/(2*pi))*diff(unwrap(angle(ftq)))*Fs];
	\end{lstlisting}
	
	Finalmente luego se procede a graficar ambos procedimientos realizados para realizar la comparación entre las señales, mostrada en la figura \ref{fig:mod-demod-fm-audio} en la cual se muestra la modulación y demodulacion de un tono con la finalidad de poder comprobar el procedimiento realizado, obteniendo para la señal demodulada una aproximación muy cercana a la señal de información.
	
	\begin{figure}[h]
		\centering
		\includegraphics[width=0.5\textwidth]{media/mod-demod-fm-audio}
		\caption{Señal Modulada y Demodulada - FM}
		\label{fig:mod-demod-fm-audio}
	\end{figure}
	
	En el bloque de código \ref{lst:modulacion-fm} tan solo es necesario definir los datos de la señal de voz en lugar del tono para realizar el proceso de modulación y su inverso para un archivo .wav.
	
	\begin{figure}[h]
		\centering
		\includegraphics[width=0.5\textwidth]{media/mod-dmod-audio}
		\caption{Modulación y Demodulación FM - Señal de Audio}
		\label{fig:mod-dmod-audio}
	\end{figure}
	
	Finalmente al realizar este cambio para la señal de voz grabada se muestra en la figura \ref{fig:mod-dmod-audio} en la cual se puede apreciar cierta fidelidad entre las señales en banda base y la señal modulada respectivamente.
	
	\bibliographystyle{IEEEtran}
	\bibliography{biblio}
\end{document}
